\section{First set up}
\subsection{Virtual Machine}
This section assumes a Azure Administrator has created a barebone Linux VM. The only task for the VM is to run a python script in regular
intervals (every minute). The set up consists of installing all required Python libraries, setting up the configuration file, and creating a 
cron job for the regular runs.

\subsubsection{Setup SSH access to VM}
SSH access to the VM is required firstly to set it up, but also to keep the VM updated. Steps:
\begin{enumerate}
  \item Log into Azure
  \item Find the VM in 'Resource Groups' or 'Virtual Machines'
  \item In the section for 'Support and Troubleshooting' select 'Reset Password'
  \item Choose 'Reset SSH public key'
  \item Choose a user name and copy your public key into 'SSH public key' field.
\end{enumerate}
For ease of use I have set up an alias in .bash\_aliases with the IP address of the VM found on the Azure portal.

To gain root access on the VM do \verb|sudo bash|.

It is advisable to run an update of the operating system before continuing to install new software.

The service VMs are set up, so that you need to open the firewall for certain time before you can ssh into them.
In the Security Centre of the Azure Portal you find the 'Advanced Cloud Defense' with an item 'Just in time VM access'. 
Find your VM in there, tick the checkbox and click the button 'Request access'; select port 22 and click 'Open ports'.
A short time late you'll gain access.

\subsubsection{Create a service account for running python}
\label{sec:useraccout}

Idealy you want an service account (\verb|useradd -r commander|) for the user running the python script in a cron job. But this was not done on any of our environments and
for consistency reasons we go for a normal user account which can't be used for login.
\begin{verbatim}
useradd commander
usermod -L commander
\end{verbatim}
This means the user still got a home directory \verb|/home/comander| and an associated shell (\verb|/bin/bash|).

\subsubsection{Install Python libraries}

Install python libraries as the user who should run them, i.e. commander. That means only this user can execute those libraries.
\begin{verbatim}
pip3 install 'azure-storage-blob==1.1.0'
pip3 install 'azure-storage-file==1.1.0'
pip3 install pymssql
#pip install crypto
pip3 install pycryptodome
pip3 install 'openpyxl==2.6.4'
pip3 install pycel
pip3 install ftpretty
pip3 install zlib
pip3 install jsonschema

yum install zlib-devel
\end{verbatim}

As an admin install

\verb|apt-get install python3-crypto|

\subsubsection{Configure python script}

The python script requires a \verb|.config| file for its configuration. This file is not part of the git repository as it contains sensitive information.
The basic structure is:
\verb|key; value|.

For the list of keys see Table~\ref{tab:table1}.

\begin{table}[ht!]
  \setlength\extrarowheight{2pt}
  \begin{center}
    \caption{Table of required keys for the configuration}
    \label{tab:table1}
    \begin{tabularx}{\textwidth}{|l|L|L|} % <-- Alignments: 1st column left, 2nd middle and 3rd right, with vertical lines in between
      \hline
      \textbf{Key} & \textbf{Description} & \textbf{Where to find}\\
      \hline
      remotehostname & DNS name of database server & Azure portal; SQL database; Overview; Server name\\
      remoteusername & User name to access the DB & Azure portal; DECMT\_DEV\_eTarget resourcegroup; eTargetVault; Secrets; find the right db \\
      remotepassword & password to remoteusername & Azure portal; DECMT\_DEV\_eTarget resourcegroup; eTargetVault; Secrets; find the right db \\
      remotedbname & Name of the database & Azure portal; SQL database; SQL databases\\
      fileuser & Username for storage where users upload data files to (i.e. the data directory) & Azure portal; Storage account; Settings; Access keys; storage account name\\
      filekey & Access key for that storage account & Azure portal; Storage account; Settings; Access keys; key1 or key2 key\\
	  patientkey & Is used to encrypt and decrypt the Christie patient number & Azure portal; DECMT\_DEV\_eTarget resource group; eTargetVault; Secrets; Encryption Key\\
      storageurl & Connection string to the storage account & Azure portal; Storage Account; Settings; Access keys; use connection string\\
      containername & Name directory in blob usually 'reports' & will be created
      if not existent\\
      logblob & Name of the blob container for log files & create with the file explorer or in the Azure portal\\
      \hline
    \end{tabularx}
  \end{center}
\end{table}

\subsubsection{Setup cron job for python script}

The python script should run as the user created in section \ref{sec:useraccout}.
\begin{verbatim}
sudo bash
su commander
crontab -e
\end{verbatim}
Add the line (adapt to the correct location):

\verb|*/1 * * * * bash /home/commander/target-data/run.sh /dev/null 2>&1|

and save. This runs the run.sh script every minute. The script itself makes sure it is not running more than once, i.e. if the ingestion takes longer than a minute.

\subsection{Database}
There is currently no SQL script for setting up an empty eTarget database. Therefore you currently have to copy an existing eTarget database 
(from the Dev or Test environment) to the new Database Service and clean out the existing data. 

\subsection{Storage}
Create a file share \verb|data| on Azure portal or via a Storage Explorer.

Create a blob share \verb|log| on Azure portal or via a Storage Explorer.

\subsection{Application}

Please read the sections \ref{sec:build}, \ref{sec:deploy}.

For a new environment add a server section as described in \ref{sec:build}.

In \verb|pom.xml| create a new profile section. See table~\ref{tab:table2} for explainations of the \verb|<properties>|.

\begin{table}[ht!]
  \setlength\extrarowheight{2pt}
  \begin{center}
    \caption{Table of required properties in pom.xml \textless profile\textgreater ~section}
    \label{tab:table2}
    \begin{tabularx}{\textwidth}{|l|L|L|} % <-- Alignments: 1st column left, 2nd middle and 3rd right, with vertical lines in between
      \hline
      \textbf{property} & \textbf{Description} & \textbf{Where to find}\\
      \hline
	  serverid & given id in .m2/settings.xml & see that file\\
	  postfix & name of the war file is \verb|etarget|+postfix\verb|.war| & set here\\
	  azureFtpUrl & in case the deploy to the azure environment happens via sftp (wagon plug-in), this is the URL used & Azure portal; App service; Overview; FTP hostname + /sites/wwwroot\\
	  webapp.include.path & where to find the include directory on the web server; note: this changes between Windows and Linux hosts & Azure portal; App service; Advanced Tools/SSH; than look for location\\
	  databaseURL & JDBC url for connecting with the DB & \makecell[l]{\verb|jdbc:jtds:sqlserver://| \\ + Server name + \verb|/| \\ + db name;  Server name \\ and db name can be \\ found in Azure portal; \\ SQL database; Overview}\\
	  dbuser & User name which the web-app should use & Azure portal; DECMT\_DEV\_eTarget resourcegroup; eTargetVault; Secrets; find the right db\\
	  dbpassword & Password for dbuser & Azure portal; DECMT\_DEV\_eTarget resourcegroup; eTargetVault; Secrets; find the right db\\
	  storageURL & Location of PDF files & Azure portal; Storage account; Access key; Connection string\\
	  \makecell[l]{RESOURCE \\ GROUP\_NAME} & Info about your Azure subscription; used for azure maven deploy & az webapp list\\
	  WEBAPP\_NAME & Name for this webapp in Azure; used for azure maven deploy & az webapp list\\
	  REGION & Region in which the webapp is deployed; used for for azure maven deploy & az webapp list\\
	  web.path & Path to the web app on the server; if on root use \textbackslash & your choice\\
      \hline
    \end{tabularx}
  \end{center}
\end{table}

\subsubsection{Configurations}

Configurations are split into three config files:
\begin{enumerate}
  \item application.properties -- contains application relevant settings
  	\begin{enumerate}
  	  \item \verb|edit.timeout| setting of the edit timeout in minutes; after that number of minutes an edit-locked patient will be released. 
  	  \item \verb|application.version| sets ver version string for the browser tab name; value taken from \verb|pom.xml|.
  	  \item \verb|application.title| Name of the application as shown on the browser tab
  	  \item \verb|application.imageURL| Footer image URLs, comma separated, either relative URLs into the eTarget application or absolute URLs to other web resources
  	  \item \verb|application.imageAlt| Alternative text for imates, displayed for screen readers or when images fail to laod
  	  \item \verb|application.support| Email address on site to request help
  	  \item \verb|data.blood| whether blood data is available
  	  \item \verb|data.tumour| whether tumour data is available
  	  \item \verb|page.additional_reports| display additional reports tab
  	  \item \verb|page.ihc| display IHC tab
  	  \item \verb|page.fmblood| display FM Bloods tab (the FM data source needs to be set-up in the \verb|CONCEPT_DATASOURCES| column \verb|panel_name| with the name `foundationmedicine')
  	  \item |verb|page.fmtumour| display FM Tumour tab (the FM data source needs to be set-up in the \verb|CONCEPT_DATASOURCES| column \verb|panel_name| with the name `foundationmedicine')
  	  \item \verb|page.tumourngs| display Tumour NGS tab
  	  \item \verb|page.ctdnangs| display CtDNA NGS tab
  	  \item \verb|page.ctdnaexploratory| display CtDNA Exploratory tab
  	  \item \verb|page.pdxcdx| display PDX CDX tab
  	  \item \verb|page.name| comma separated list of providers as found in the eTarget database table \verb|CONCEPT_DATASOURCES| column \verb|panel_name|. 
  	  Note these are the generic providers, FM Tumour and  ctDNA providers are not configures this way, just turned on with the boolean above.
  	  \item \verb|page.name.[name1]| name which should be displayed for this provider
  	  \item \verb|page.name.[name1].blood| whether blood tab should be displayed
  	  \item \verb|page.name.[name1].tumour| whether tumour tab should be displayed
  	  \item \verb|page.name.[name1].code| two letter code which will be used for this provider in reports 
	\end{enumerate}
  \item config.properties -- contains connection settings
    \begin{enumerate}
      \item \verb|serverid| Maven setting in \verb|.m2/settings.xml| which contains the login data (username + password) the server
      \item \verb|azureFtpUrl| The URL to use in case deployment is done using FTP (caused problems and generally moved to azure maven deploy)
      \item \verb|toWarFile| The location for which the war file is to be build; needs to match where to deploy in Apache
  	  \item \verb|databaseURL| URL to the database; value taken from \verb|pom.xml|. 
  	  \item \verb|dbuser| user name for the database; taken from \verb|pom.xml|.
  	  \item \verb|dbpassword| password for dbuser; taken from \verb|pom.xml|.
  	  \item \verb|storageURL| URL to Azure blob storage (location of the PDFs); taken from \verb|pom.xml|.
  	  \item \verb|storageContainerName| Name of the blob 'folder' for reports (PDFs)
  	  \item \verb|trialContainerName| Name of the blob 'folder' for trial HTML documents (generated from the trail finder)
  	  \item \verb|webapp.include.path| location of the include folder on the web server; taken from \verb|pom.xml|.
  	  \item \verb|RESOURCEGROUP_NAME| Name of the Azure resource group for Maven Azure deploy; taken from Azure portal
  	  \item \verb|WEBAPP_NAME| Name of the Azure web application; taken from Azure portal
  	  \item \verb|REGION| Name of the deploy region; note: do not take from the Azure portal, required is the short name \verb|az| client provides, the one without spaces and catital letters.
  	  \item \verb|PRICING| Pricing of the web application; taken from the Azure portal
  	  \item \verb|SUBSCRIPTION_ID| Id of the subscription; taken from the Azure portal
  	  \item \verb|web.path| location of where it should be deployed
	\end{enumerate}
  \item logging.properties -- standard logging properties file; set logging levels
\end{enumerate}

\subsection{Active Directory}

Authentication is done using Microsoft's Active Directory service. This needs to be configured on the Azure portal. This part is best done by the administrator of Azure.

Enable Authentication: Azure Portal: go to the correct App Service; Authentication/Authorization; switch on 'App Service Authentication'.\\ 
Use Azure Active Directory: 'Action to take when request is not authenticated': 'Log in with Azure Active Directory'.\\
'Authentication Providers': configure Azure Active Directory. Management Mode:
Advanced. \\

Client ID: Copy from Azure portal; Azure Active Directory; App registration; Application ID\\
Issuer URL: \verb|https://sts.windows.net/|+ Azure portal; Azure Active Directory;  Properties; Directory ID\\
 